\section{Related work}
\label{sec:related_work}
% This Section addresses existing contributions by examining the current state of architecture in the I4.0 domain. Our primary focus centers on assessing the attainment of specific Quality Attributes, namely Availability, Deployability, and Interoperability, and the associated trade-offs involved in their implementation.
% In total, 9 papers are investigated. 

% In \cite{Wan2019Reconfigurable}, experiences are elaborated on a three-layer architecture of a reconfigurable smart factory for drug packing in healthcare I4.0. 


% The paper \cite{Yazen2010Ontology} proposes an ontology agent-based architecture for inferring  new configurations to adapt to changes in manufacturing requirements and/or environment.



% In \cite{Leitao2016Specification,Angione2017Integration} an architecture for a reconfigurable production system is specified.
% Two objectives for reconfiguration and how they can be reached are described.


% Several papers \cite{Koren1999Reconfigurable,Koren2010Design,Bortolini2018Reconfigurable} describe reconfigurable manufacturing systems that are cost-effective and responsive to market changes.

% All contributions provide valuable knowledge about reconfiguration but lack a study of the software architecture perspective that specifies a quantifiable reconfigurability architectural requirement, a software architecture that adopts the architectural requirements, and evaluates the architectural requirement. 

This section addresses existing contributions by examining the current state of architecture in the I4.0 domain. Our primary focus centers on assessing the attainment of the specific Quality Attributes that this research paper focuses on. Those being Availability, Deployability, and Interoperability, and the associated trade-offs involved in their implementation.

In total, nine papers are investigated and are referred to in this paper with numbers. These papers offer valuable insights into various aspects of production system architecture and its alignment with the critical Quality Attributes mentioned above. Notably, they provide a foundation for understanding the existing gaps in research with respect to the requirements of a production system, which include high availability, continuous deployability, and interoperability.

Interoperability is an important component of a production system. Two exemplary works, namely paper \cite{Ungurean2020-nq} and paper \cite{Jepsen2021-dx}, dives deep into the challenges and solutions concerning interoperability within the context of Industry 4.0. Paper \cite{Ungurean2020-nq} focuses on the importance of standardized communication protocols and data structures in achieving interoperability. It emphasizes the role of XML-based address spaces and the use of the publisher-subscriber paradigm in facilitating versatile interactions between physical and virtual entities. On the other hand, paper \cite{Jepsen2021-dx} extends the discussion of interoperability to encompass multiple levels, ranging from technical and syntactic to semantic and pragmatic. Furthermore, they propose handling reconfiguration through the middleware either by having the knowledge distributed on each service, or incorporating the knowledge in a more centralized way. These works provide a foundational understanding of interoperability but tend to be centered around this aspect, overlooking other vital requirements, such as continuous deployability and high availability within a production system.

Continuous deployability is equally essential for modern production systems. It refers to the system's ability to adapt swiftly to changes, updates, or entirely new configurations, which are indispensable in today's dynamic manufacturing environments. Paper \cite{Jepsen2021-nq} emphasizes the need for a reconfigurable middleware architecture to support flexible production systems, detailing aspects of the reconfiguration cycle time, timeliness, and process traceability. Paper \cite{Alejandro2019-nq} contributes by examining a microservice-oriented architecture, highlighting the benefits of networked data acquisition and scalability. However, the existing literature generally lacks in-depth discussions regarding the architectural considerations needed for continuous deployability in a production system.

The aspect of ensuring high availability is fundamental, especially in manufacturing, where any disruption can have substantial consequences. Unfortunately, this fundamental aspect is often underrepresented in existing research. Even architectural frameworks that emphasize reconfigurability or event-driven approaches, as observed in works such as paper \cite{Jepsen2023-zz} and paper \cite{Theorin2017-nq}, tend to bypass the nuances of ensuring high availability. Paper \cite{Jepsen2023-zz} introduces an Event-Driven Architecture (EDA) featuring loose coupling and a prototype-oriented information model, which supports real-time monitoring, control, optimization, and reconfiguration. \cite{Theorin2017-nq} extends the discussion by focusing on the extreme loose coupling of EDA, which allows for applications to be developed and tested in isolation, promoting scalability and ease of deployment. However, these works tend to lack in-depth insights into the specific architectural mechanisms required to ensure high availability, which is a significant gap in the research considering its critical role in maintaining uninterrupted production processes in manufacturing.

In contrast, the comprehensive study conducted by \cite{Kang2016-nq} highlights the key technologies relevant to I4.0, showcasing their interconnected and complementary nature. This research emphasizes the significance of technologies like Cyber-Physical Systems, IoT, Cloud Computing, Big Data and Sensors in the I4.0 landscape. These 5 technologies are key for I4.0, while Additive Manufactoring, Energy Saving and Holograms also showed some implementation in the I4.0 domain. While \cite{Kang2016-nq} provides a valuable high-level overview of these technologies, it primarily focuses on their interconnectivity and is limited in delving deeply into how they align with the specific requirements of a production system, which necessitates high availability, continuous deployability, and interoperability.

These studies, while indispensable in their respective domains, often do not provide a comprehensive perspective on production system architecture, with a focus on high availability, continuous deployability, and interoperability while also discussing the architectural trade-offs. These attributes are vital requirements in contemporary manufacturing environments. Our study aims to address these gaps, offering guidance for designing production systems that excel in these demands while ensuring uninterrupted and efficient operations in the manufacturing industry.